\documentclass[t,aspectratio=169,autodetect-engine,ja=standard]{beamer}

\usepackage{luatexja}
\usepackage{tikz}

\usetikzlibrary{matrix}

\usetheme{Rochester}
\usecolortheme{rose}
\usefonttheme{professionalfonts}
\setbeamertemplate{footline}[frame number]

\renewcommand\kanjifamilydefault{\gtdefault}
\renewcommand\familydefault{\sfdefault}
\usefonttheme[onlymath]{serif}

\renewcommand{\baselinestretch}{1.3}

\title{アルゴリズムとデータ構造}
\author{終に鮭}
\institute{電子計算機研究会}
\date[2022/7/xx]{2022年7月吉日}

\begin{document}
\frame{\titlepage}

\begin{frame}{概要}
\begin{itemize}
    \item 授業ではフォーカスされないデータ構造とアルゴリズムを紹介
    \item 座学形式
        \begin{itemize}
            \item サンプルプログラムは配布
        \end{itemize}
    \item 今回はバケットソート・基数ソート
\end{itemize}
\end{frame}

\section{バケットソート}
\frame{\sectionpage}

\begin{frame}{バケットソート}
\begin{itemize}
    \item 非比較ソートアルゴリズム(安定ソート)
    \item 配列のランダムアクセス
    \item 2つバリエーションがある
        \begin{itemize}
            \item 計数ソート(これ)
            \item キューを用いた実装
        \end{itemize}
    \item ソートする配列の長さ $N$、キーの整数値の範囲 $[0,R)$、要素のサイズ $S$ に対し
        \begin{itemize}
            \item 時間計算量 $O(N+R)$
                \begin{itemize}
                    \item これは比較ソートの $O(N \log N)$ と比べて高速
                \end{itemize}
            \item 空間計算量 $O(NS+R)$
        \end{itemize}
\end{itemize}
\end{frame}

\begin{frame}{バケットソートの図解}
    \centering
    \begin{tikzpicture}
        \matrix[
            ampersand replacement=\&
        ] (unsorted) {
            \alt< 3,27,28>{\node[fill=yellow]{6}}{\node{6}}; \&
            \alt< 4,29,30>{\node[fill=yellow]{0}}{\node{0}}; \&
            \alt< 5,31,32>{\node[fill=yellow]{0}}{\node{0}}; \&
            \alt< 6,33,34>{\node[fill=yellow]{6}}{\node{6}}; \&
            \alt< 7,35,36>{\node[fill=yellow]{7}}{\node{7}}; \&
            \alt< 8,37,38>{\node[fill=yellow]{4}}{\node{4}}; \&
            \alt< 9,39,40>{\node[fill=yellow]{6}}{\node{6}}; \&
            \alt<10,41,42>{\node[fill=yellow]{2}}{\node{2}}; \&
            \alt<11,43,44>{\node[fill=yellow]{1}}{\node{1}}; \&
            \alt<12,45,46>{\node[fill=yellow]{1}}{\node{1}}; \&
            \alt<13,47,48>{\node[fill=yellow]{7}}{\node{7}}; \&
            \alt<14,49,50>{\node[fill=yellow]{3}}{\node{3}}; \\
        };
        \matrix<2-15>[
            ampersand replacement=\&
        ] (count0) at (0,-1) {
            \node<-3>{0}; \node<4>[fill=orange]{1}; \alt<5>{\node[fill=orange]{2}}{\node<5->{2}}; \&
            \node<-10>{0}; \node<11>[fill=orange]{1}; \alt<12>{\node[fill=orange]{2}}{\node<12->{2}}; \&
            \node<-9>{0}; \alt<10>{\node[fill=orange]{1}}{\node<10->{1}}; \&
            \node<-13>{0}; \alt<14>{\node[fill=orange]{1}}{\node<14->{1}}; \&
            \node<-7>{0}; \alt<8>{\node[fill=orange]{1}}{\node<8->{1}}; \&
            \node{0}; \&
            \node<-2>{0}; \alt<3>{\node[fill=orange]{1}}{\node<3-5>{1}}; \alt<6>{\node[fill=orange]{2}}{\node<6-8>{2}}; \alt<9>{\node[fill=orange]{3}}{\node<9->{3}}; \&
            \node<-6>{0}; \alt<7>{\node[fill=orange]{1}}{\node<7-12>{1}}; \alt<13>{\node[fill=orange]{2}}{\node<13->{2}}; \\
        };
        \node<2-15> [align=center] at (0,-2) {
            キー(=値)の範囲は $[0,8)$ →長さ8の配列を用意 \\
            キーの出現回数をカウント
        };
        \matrix<16->[
            ampersand replacement=\&
        ] (count) at (0,-1) {
            \alt<18>{\node[fill=yellow]{2}}{\node{2}}; \&
            \alt<19>{\node[fill=yellow]{2}}{\node{2}}; \&
            \alt<20>{\node[fill=yellow]{1}}{\node{1}}; \&
            \alt<21>{\node[fill=yellow]{1}}{\node{1}}; \&
            \alt<22>{\node[fill=yellow]{1}}{\node{1}}; \&
            \alt<23>{\node[fill=yellow]{0}}{\node{0}}; \&
            \alt<24>{\node[fill=yellow]{3}}{\node{3}}; \&
            \node{2}; \\
        };
        \matrix<16-25>[
            ampersand replacement=\&
        ] (insertpos) at (0,-2) {
            \node<-16>{*}; \node<17>[fill=orange]{0}; \node<18>[fill=yellow]{0}; \node<19->{0}; \&
            \node<-17>{*}; \node<18>[fill=orange]{2}; \node<19>[fill=yellow]{2}; \node<20->{2}; \&
            \node<-18>{*}; \node<19>[fill=orange]{4}; \node<20>[fill=yellow]{4}; \node<21->{4}; \&
            \node<-19>{*}; \node<20>[fill=orange]{5}; \node<21>[fill=yellow]{5}; \node<22->{5}; \&
            \node<-20>{*}; \node<21>[fill=orange]{6}; \node<22>[fill=yellow]{6}; \node<23->{6}; \&
            \node<-21>{*}; \node<22>[fill=orange]{7}; \node<23>[fill=yellow]{7}; \node<24->{7}; \&
            \node<-22>{*}; \node<23>[fill=orange]{7}; \node<24>[fill=yellow]{7}; \node<25->{7}; \&
            \node<-23>{*}; \node<24>[fill=orange]{10}; \node<25->{10}; \\
        };
        \node<16-25> [align=center] at (0,-4) {
            長さ8の配列を用意 \\
            $i$ 番目の要素を $[0,i)$ の出現回数の総和にする \\
            →ソート後の配列におけるキー $i$ の\textcolor{blue}{最初の出現位置}に相当
        };
        \matrix<26->[
            ampersand replacement=\&
        ] (insertpos) at (0,-2) {
            \node<-28>{0}; \node<29>[fill=yellow]{0}; \node<30>[fill=orange]{1}; \node<31>[fill=yellow]{1}; \alt<32>{\node[fill=orange]{2}}{\node<32->{2}}; \&
            \node<-42>{2}; \node<43>[fill=yellow]{2}; \node<44>[fill=orange]{3}; \node<45>[fill=yellow]{3}; \alt<46>{\node[fill=orange]{4}}{\node<46->{4}}; \&
            \node<-40>{4}; \node<41>[fill=yellow]{4}; \alt<42>{\node[fill=orange]{5}}{\node<42->{5}}; \&
            \node<-48>{5}; \node<49>[fill=yellow]{5}; \alt<50>{\node[fill=orange]{6}}{\node<50->{6}}; \&
            \node<-36>{6}; \node<37>[fill=yellow]{6}; \alt<38>{\node[fill=orange]{7}}{\node<38->{7}}; \&
            \node{7}; \&
            \node<-26>{7}; \node<27>[fill=yellow]{7}; \alt<28>{\node[fill=orange]{8}}{\node<28-32>{8}}; \node<33>[fill=yellow]{8}; \alt<34>{\node[fill=orange]{9}}{\node<34-38>{9}}; \node<39>[fill=yellow]{9}; \alt<40>{\node[fill=orange]{10}}{\node<40->{10}}; \&
            \node<-34>{10}; \node<35>[fill=yellow]{10}; \alt<36>{\node[fill=orange]{11}}{\node<36-46>{11}}; \node<47>[fill=yellow]{11}; \alt<48>{\node[fill=orange]{12}}{\node<48->{12}}; \\
        };
        \matrix<26-50>[
            ampersand replacement=\&
        ] (sorted0) at (0,-3) {
            \node<-28>{*}; \alt<29>{\node[fill=orange]{0}}{\node<29->{0}}; \&
            \node<-30>{*}; \alt<31>{\node[fill=orange]{0}}{\node<31->{0}}; \&
            \node<-42>{*}; \alt<43>{\node[fill=orange]{1}}{\node<43->{1}}; \&
            \node<-44>{*}; \alt<45>{\node[fill=orange]{1}}{\node<45->{1}}; \&
            \node<-40>{*}; \alt<41>{\node[fill=orange]{2}}{\node<41->{2}}; \&
            \node<-48>{*}; \alt<49>{\node[fill=orange]{3}}{\node<49->{3}}; \&
            \node<-36>{*}; \alt<37>{\node[fill=orange]{4}}{\node<37->{4}}; \&
            \node<-26>{*}; \alt<27>{\node[fill=orange]{6}}{\node<27->{6}}; \&
            \node<-32>{*}; \alt<33>{\node[fill=orange]{6}}{\node<33->{6}}; \&
            \node<-38>{*}; \alt<39>{\node[fill=orange]{6}}{\node<39->{6}}; \&
            \node<-34>{*}; \alt<35>{\node[fill=orange]{7}}{\node<35->{7}}; \&
            \node<-46>{*}; \alt<47>{\node[fill=orange]{7}}{\node<47->{7}}; \\
        };
        \node<26> [align=center] at (0,-4) {
            長さ $12(=N)$ の配列を用意 \\
            ここにソート結果を格納する
        };
        \node<27-50> [align=center] at (0,-4) {
            もとの配列を順に走査しながら前の手順で作った配置場所の配列を使って整列 \\
            配置場所配列の使った要素はインクリメント
        };
        \matrix<51>[matrix of nodes, ampersand replacement=\&, draw=blue] (sorted) at (0,-3) { 0 \& 0 \& 1 \& 1 \& 2 \& 3 \& 4 \& 6 \& 6 \& 6 \& 7 \& 7 \\ };
        \node<51> [align=center] at (0,-4) {
            ソート完了
        };
    \end{tikzpicture}
\end{frame}

\begin{frame}{バケットソートの長短}
    \begin{itemize}
        \item 長所
            \begin{itemize}
                \item 極めて高速(時間計算量 $O(N+R)$)
                \item 実装が単純
            \end{itemize}
        \item 短所
            \begin{itemize}
                \item 整数しかソートできない
                \item キーの範囲が広いと使えない
            \end{itemize}
    \end{itemize}
\end{frame}

\end{document}
